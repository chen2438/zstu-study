\section{总结与展望}
\subsection{主要研究成果和结论}
本文通过调研地库排布问题,分析当前车位自动化排布存在的问题,重新提出一种基于PER-D3QN的车位排布算法,并通过多个案例,验证算法的有效性。本文主要研究内容如下:
\begin{enumerate}
    \item 本文深入研究了地库车位排布问题,发现当前研究存在着一定的局限性,例如奖励机制、过度依赖人工干预、出入口位置后定等,因此提出一种基于智能体行为导向和PER-DQN的车位排布算法,来解决该类问题。
    \item 针对车位放置问题,提出了一种创新的基于智能体行为的车位布局方法。该方法依据智能体的行为(如直行或转弯)以及其所处位置来决定车位的布置方式。同时,还考虑了环境中障碍物对车位布局的影响,并据此提出了相应的布局策略。此外,本文还研究了柱网布置,这是车位布局中的一个重要因素。
    \item 本文设计了一个神经网络模型,该模型能够处理输入的状态信息,并输出每个行为的Q值。该模型包括数据初始化、特征提取网络、特征融合网络,有效地处理和理解输入的状态信息,为智能体的决策提供更准确的依据。此外,模型还包括一个噪声网络层,以增强模型的探索能力。通过向模型的输出添加一些随机噪声,模型可以探索更多的可能性,从而提高车位布局的效率和质量。
    \item 本文将车位布局过程抽象为一个强化学习问题,定义了状态空间、行动空间以及智能体的奖励机制,构建了一个基于PER-D3QN的强化学习模型。设计了一个奖励评价系统,用于评估每一轮算法道路布置的效果,其中包括两部分:车辆和交通。车辆部分主要考虑车位数量的变化,交通部分则主要反映当前的道路铺设情况,包括直路奖励、道路过宽惩罚和重复铺设惩罚。
    \item 为了验证本文算法的有效性,本文使用了6张CAD工程图纸,与王潇霆使用的算法进行对比,证明了实验的有效性。
\end{enumerate}    

基于上述内容,并对比实验结果,本文得出以下结论:
\begin{enumerate}
    \item 通过边铺设道路边排布车位的方式,可以直接通过排布情况给予智能体实时反馈,从而提高车位排布的效率。
    \item 本文首次在具有复杂边界和多障碍群的图纸中引入了基于智能体行为导向的车位排布,使得强化学习真正应用到车位排布问题,而非仅用于道路铺设。
    \item 通过减少人工干预,本文能够让智能体更好地学习和适应环境,提高了算法的决策能力和精确性。
    \item 同时,本文采用了PER-D3QN来优化奖励较低的个体,并利用网络来更深入地解析周围信息,以辅助智能体的理解和决策。
\end{enumerate}
\subsection{展望}
本文提出的基于智能体行为导向和PER-D3QN的车位排布算法,虽然在实验中取得了一定的效果,但仍然存在一些问题和不足之处,需要进一步完善和改进:
\begin{enumerate}
    \item 本研究的模型以单一出入口的停车场设计为基础。然而,实际的大型停车场设计通常需要2-3个出入口以提升车流的流动性和安全性。因此,未来的研究将扩展此模型以适应多出入口的场景。
    \item 在本研究的模型中,智能体的行走方式主要是垂直和水平的。为了更好地适应边界斜边,未来的研究将探索引入多角度的行走方式。
    \item 本研究的模型在用户体验方面还有待提升,例如,用户可能需要绕行多次才能找到车位。未来的研究将进一步优化智能体的行走策略,以提高停车场的导航效率和用户的停车体验。
\end{enumerate}
% \thispagestyle{empty}
\newgeometry{left=0.98in,right=0.98in,top=0.98in,bottom=0.98in}
\begin{center}
    \heiti \zihao{3}浙江理工大学本科毕业设计(论文)开题报告
\end{center}

\begin{table}[H]
    \centering
    \begin{tblr}{
        colspec = {|X[0.65in,c]|X[2.2in,c]|X[0.8in,c]|X[0.8in,c]|X[1.43in,c]|},
        hline{1,2,3,4,6,8}={solid},
        rowsep=0pt,
        colsep=0.05in,
        row{1,2} = {0.5in},
        row{3} = {3.6in},
        row{4} = {2.0in},
        row{5} = {0.5in},
        row{6} = {0.9in},
        row{7} = {0.3in}
    }
        \textbf{班\quad 级} &  启新卓越实验班20(1)  &
        \SetCell[c=2]{c} \textbf{姓\qquad 名} & & 孙斓绮 \\
        \textbf{课题名称} & \SetCell[c=4]{c}{基于机器学习的车位规划} \\ 
        \SetCell[c=5]{l,h}{
            \textbf{开题报告}(包括以下5点内容,(艺术类、外语类除外)不少于3000字)\\
            1、选题意义与可行性分析\\
            2、研究的基本内容与拟解决的主要问题\\
            3、总体研究思路及预期研究成果\\
            4、研究工作计划\\
            5、参考文献

            \vspace{1cm}

            \qquad (本页列出开题报告目录,报告全文附后。)

            \vspace{1cm}

            \qquad \textbf{成绩:}
        }\\
        \SetCell[r=2]{m,c}\textbf{开题答辩\\意见} & \SetCell[c=4]{l,h}{
            (选题意义、工作量和难易度评价,是否同意开题,100字以上)
        } \\ 
        & \SetCell[c=4]{r,f}{
            答辩组长签名: \qquad \qquad \qquad \qquad

            \vspace{1ex}
            年 \hspace{2em} 月 \hspace{2em} 日
        } \\
        \SetCell[r=2]{m,c}\textbf{指导老师\\知情确认} & \SetCell[c=1]{l,f}{
            签名:
        } & 
        \SetCell[r=2]{m,c}\textbf{系主任\\审核\\意见} & \SetCell[c=2]{l,f}{
            签名:
        } \\ 
        & \SetCell{r,f}{
            年 \hspace{2em} 月 \hspace{2em} 日
        } & & \SetCell[c=2]{r,f}{
            年 \hspace{2em} 月 \hspace{2em} 日
        } \\
    \end{tblr}
\end{table}

\restoregeometry